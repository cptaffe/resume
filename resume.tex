
% Imports and shit

\documentclass{resume} % Use the custom resume.cls style

\usepackage[left=0.75in,top=0.6in,right=0.75in,bottom=0.6in]{geometry} % Document margins
\usepackage{hyperref}
\hypersetup{
	colorlinks=true,
	urlcolor=blue,
}

% add tabular command

\newcommand{\btab}[2]{
	\bgroup
	\def\arraystretch{#1}
	\begin{tabular}{#2}
}

\newcommand{\etab}{
	\end{tabular} \smallskip
	\egroup
}

\name{Connor Taffe} % Your name
\address{3101 S. Taylor St. \\ Little Rock, Arkansas 72204} % Your address
\address{(501)~$\cdot$~606~$\cdot$~1807 \\ cpaynetaffe@gmail.com} % Your phone number and email

\begin{document}

% Education Section

\begin{rSection}{Education}

\begin{rSubsection}{University of Arkansas, Little Rock}{June 2018}{B.S. in Computer Science}

	\item Courses: Data Structures and Algorithms, Computer Systems and Assembly Language, Calculus II, Language Structure, Databases, and Discrete Math.
\end{rSubsection}

% hspace{0.0em} somehow fixes the margin
{\bf Mississippi School for Mathematics and Sciences} \hfill {2012 - 2013} \hspace{0.0em}

\begin{rSubsection}{Arkansas School for Mathematics, Sciences, and the Arts}{2013 - 2014}{}

	\item First place Intel International Science and Engineering Fair project in Materials Engineering at the local level on my research into 3D printing large angles.
	\item Courses: Computer Programming II
\end{rSubsection}

% hspace{0.0em} somehow fixes the margin
{\bf Cabot High School} \hfill {2014} \hspace{0.0em}

\end{rSection}

% Work Experience
\begin{rSection}{Experience}

% Emerging Analytics Center
\begin{rSubsection}{Emerging Analytics Center, UALR}{October 2014 - Current}{Software Engineering Intern}{Little Rock, AR}

	\item Developed Data Visualization solutions for Oculus Rift, CAVE system
	\item Used Unity 3D (scripting with C\#) for 3D programming and model manipulation
\end{rSubsection}

% Research Assistant
\begin{rSubsection}{BioInformatics}{August - October 2014}{Research Assistant}{Little Rock, AR}

	\item Refactored Genetic algorithm code with an emphasis on common stylistic guidelines, and concurrency
\end{rSubsection}

% Future Tek
\begin{rSubsection}{Future Tek Inc.}{June - August 2012}{Contract Graphic Designer}{Columbus, MS}

	\item Worked as a contract graphic designer to produce a new catalog, logo, and social media presence. All graphics were designed personally with Inkscape and Gimp
	\item This catalog is still in use (\href{http://www.futuretekinc.com/wp-content/uploads/2014/08/Future-Tek-Catalog.pdf}{link})
\end{rSubsection}

% Mesher freelance
\begin{rSubsection}{\href{http://github.com/cptaffe/Mesher}{Mesher}}{July - August 2014}{Freelance Developer}{}

	\item Client side stl-editing application built atop THREE.js, a popular WebGL interface
	\item Provides a clean interface to many of the common operations done on STL models before 3D printing
	\item This application provides a interactable 3D interface for models generated from STL files
	\item Provides multiple modification and save options
\end{rSubsection}

% CentOS freelance
\begin{rSubsection}{CentOS Server Wordpress Installation}{August 2014}{Freelance Developer}{}

	\item Installed Wordpress on a pre-imaged CentOS server.
	\item Set up a mysql database and edited Wordpress configs.
\end{rSubsection}

\end{rSection}

\clearpage

% Personal Projects
\begin{rSection}{Personal Projects}

% Lispy project
\begin{rSubsection}{\href{http://github.com/cptaffe/lispy}{Lispy}}{September - November 2014}{}{}

	\item Lisp-like interpreter written in Python
	\item Producer-consumer threading optimizes stages.
	\item Lazy-evaluation of defined variables
	\item lambda functions
	\item recursive lambdas
	\item EBNF formal definition of the language.
\end{rSubsection}

% Tcpwsh project
\begin{rSubsection}{\href{http://github.com/cptaffe/tcpwsh}{TcpWsh}}{November 2014}{}{}

	\item Shell script which opens tcpdump with root privileges and routes packets using an unnamed FIFO to wireshark running with standard permissions.
	\item Prevents wireshark from being exploited remotely
	\item Allows utilization of tcpdump for efficient in-kernel packet filtering.
\end{rSubsection}

\end{rSection}

% Technical Strengths

\begin{rSection}{Technical Strengths}

{\bf General Programming}

\btab{1.1}{ l l }
	Languages: & C++, C, Go, Java, Python, C\#, Rust \\
	Version Control Systems: & git \\
	Networking: & C, Go, Java, Python \\
\etab

{\bf Operating Systems}

\btab{1.1}{ l l }
	BSD: & OpenBSD, FreeBSD, Dragonfly BSD, OS X \\
	Linux: & Debian, Red Hat Linux, CentOS, Fedora, Ubuntu, Manjaro \\
	Windows: & 2000, XP, 7, 8
\etab

{\bf Unix System Administration}

\btab{1.1}{ l l }
	Proficient in shell: & bash, sh, tcsh \\
	Proficeint with utils: & grep, sed, ssh, vim, nano, crontab, tcpdump, tar etc. \\
	Experience with: & configuration files, package managers, compilation \\
	Familiar with *nix concepts: & users, permissions, processes, cron jobs, files and i/o redirection, etc.
\etab

{\bf Web Programming}

\btab{1.1}{ l l }
	Server side: & Python, Go, Java, Php, and SQL (MySQL) \\
	Client side: & JavaScript (JQuery, THREE.js, etc.), CSS, (X)HTML \\
	Servers: & Apache, nginx, lighttpd \\
	Written Servers in: & Python, Go \\
\etab

\end{rSection}

\end{document}
